\chapter*{Introduction générale }
\addcontentsline{toc}{chapter}{Introduction générale}\label{chap.intro}
L'avènement de l'Industrie 4.0 a propulsé la transformation numérique des industries, en introduisant des technologies innovantes qui redéfinissent la façon dont nous concevons, construisons et exploitons les systèmes physiques. Au cœur de cette révolution se trouve la technologie des jumeaux numériques (DT), qui offre une représentation virtuelle complète et dynamique d'un actif physique. Les DT permettent une surveillance en temps réel, une optimisation des performances et une prise de décision proactive, transformant ainsi divers secteurs industriels.

Ce mémoire explore les fondements théoriques et les applications pratiques des DT, en mettant l'accent sur leur rôle dans l'amélioration de l'efficacité, de la fiabilité et de la durabilité des systèmes physiques. Nous examinons les technologies habilitantes qui sous-tendent les DT, telles que l'Internet des objets (IoT) et la blockchain, et explorons les travaux de recherche existants sur la sécurité des DT. De plus, nous discutons des défis et des opportunités associés à l'adoption des DT, en tenant compte des implications de sécurité, de confidentialité et d'éthique.
Le mémoire s'articule autour de trois chapitres principaux:

Chapitre 01: Introduction aux Jumeaux Numériques (DT)
Ce chapitre présente les concepts fondamentaux des DT, en décrivant leur architecture, leurs avantages et leurs cas d'utilisation potentiels. Il explore également l'historique du développement des DT et leur évolution dans le contexte de l'Industrie 4.0.

Chapitre 02: Blockchain et IoT pour les Jumeaux Numériques Ce chapitre se concentre sur deux technologies clés qui jouent un rôle crucial dans l'implémentation et la sécurisation des DT: la blockchain et l'IoT. Il examine comment la blockchain peut garantir l'immuabilité, la transparence et la traçabilité des données des DT, tandis que l'IoT fournit les données nécessaires pour alimenter les modèles numériques.

Chapitre 03: Sécurité des Jumeaux Numériques: Travaux de Recherche Existants
Ce chapitre explore les défis de sécurité uniques posés par les DT et examine les travaux de recherche existants visant à les aborder. Il présente des approches et des solutions de sécurité pour protéger les données sensibles, garantir l'intégrité des systèmes et atténuer les cybermenaces.
