
\begin{abstract}
    \thispagestyle{plain}
    \pagenumbering{roman}
    \setcounter{page}{2}
    \phantomsection\addcontentsline{toc}{chapter}{\abstractname}


    Aphasia is a language disorder caused by brain damage (most commonly a stroke).
    Broca's aphasia is a form of aphasia that impairs language production.
    It is caused by an injury to Broca's Area, an area of the frontal lobe of the brain; responsible for language decoding.
    A person suffering from Broca's aphasia may find it difficult to articulate words and sentences.
    However, they generally can understand what is said to them.
    This form of aphasia is associated with a lower quality of life and a higher risk of depression and suicide.

    Speech therapy is the most commonly prescribed remedy to people with Broca's aphasia.
    Despite its effectiveness, it remains an expensive, time-consuming, and effort-heavy process.
    This makes it inaccessible to a significant number of people with aphasia.

    The use of natural language processing-based techniques to improve these people's quality of life
    is an emerging research avenue that has enjoyed the attention of many researchers in recent years.

    In this graduation project,
    we are interested in the use of machine translation and automatic speech recognition
    to partially automate the rehabilitation of people with aphasia.

    To this end, we conduct a bibliographic study in which introduce aphasia, its causes, consequences,
    the problems of classical treatment methods, and a literature review the existing works
    pertaining to machine translation and automatic speech recognition.

    We then design a system that corrects french aphasic speech
    by combining a translation model with a speech recognition model, the former of which we implement.
    We finish by presenting the results of our work:
    a corpus for automatic speech recognition,
    and a translation model with a {bleu} score of \(79.61\%\).
    %%
    \\ [2cm]
    %%
    \rule{\linewidth}{1pt}

    \textbf{Keywords --- } Broca aphasia, Machine learning, Natural language processing, Machine translation, Automatic speech recognition, Transformer.\\
    \rule{\linewidth}{1pt}
\end{abstract}


\begin{otherlanguage}{french}
    \begin{abstract}
        \thispagestyle{plain}
        \pagenumbering{roman}
        \setcounter{page}{1}
        \phantomsection\addcontentsline{toc}{chapter}{\abstractname}

        Ceci est un résumé en français.

        %%
        \ \\[2cm]
        %%
        \rule{\linewidth}{1pt}

        \textbf{Mots clés --- } Aphasie de Broca, Transformeur.\\
        \rule{\linewidth}{1pt}
    \end{abstract}

\end{otherlanguage}
\renewcommand{\abstractname}{\RL{مـلـخـص}}
\afterpage{
    \newgeometry{top=-2cm}
    \begin{abstract}
        \thispagestyle{plain}
        \pagenumbering{roman}
        \setcounter{page}{3}
        \phantomsection\addcontentsline{toc}{chapter}{\textRL{مـلـخـص}}
        \begin{RLtext}
            الـحـبـسـة إضـطـرابٌ لـغـوي نـاتـج عن تـلـف فـي الـدمـاغ، غـالـبـا نـتـيـجـة سـكـتـة دمـاغـيـة.
            حـبـسـة بـروكـا حـبـسـة تـنـتـج عـن إصـابـة فـي مـنـطـقـة بـروكـا،
            وهـي مـنـطـقـة فـي الـفـص الجـبـهي الأيـسـر للـدمـاغ تـعـنـى بـإنـتـاج الكـلام.
            قـد يـجـد الـمـصـاب بـحـبـسـة بـروكـا صـعـوبـة فـي تـكـويـن الـجـمـل والـكـلـمـات،
            إلا أنـه عـادة يـفـهـم مـا يـقـال.
            تـرتـبـط هـذه الحـبـسـة بـتـدنـي مـسـوى الـعيـش وارتـفتاع  خـطر الاكـتـئـاب والانـتـحـار.

            عـلاج الـنـطق هـو أكـثـر الـعـلاجـات وصـفـا للمـصـابـيـن بـحـبـسـة بـروكـا.
            رغـم نـجـاعـتـه، فـهو يـظـل مـكـلـفـا للـوقـت والـمـال والـجـهـد،
            مـا يـحـول دون تـوفـره لعـدد كـبـيـر مـمـن يـحـتـاجـونـه.

            تـوظـيـف تـقـنـيـات مـعـالـجـة اللـغـة الـطـبـيـعـيـة لـتـحـسـيـن حـيـاة الـمـصـابـيـن بـحـبـسـة بـروكـا
            مـجـال بـحـث حـظي بـاهـتـمـام الـعـديـد مـن الـبـاحـثـيـن فـي الأعـوام الأخـيـرة.

            فـي مـشـروع الـتـخـرج هـذا، نـهـتـم بـاسـتـعـمـال الـتـرجـمـة الآلـية والـتـعرف الآلـي عـلى الكـلام
            لتـأديـة جـزء مـن عـلاج الـنـطق لـحـبـسـة بـروكـا أوتـومـاتـيـكـيـا.
            مـن أجـل ذلك، نـعـرض دراسـة بـيـبـلـيـوغـرافـيـة
            نـعـرف فـيـهـا بـحـبـسـة بـروكـا أسـبـابـا ونـتـائـج،
            ثـم نـتـطرق لـعـيـوب الـعـلاجـات الـمـعـتـادة.
            وللأعـمـال الـتـي سـبـق إنـجـازهـا فـي مـجـالـي الـتـرجـمـة الآلـية والـتـعرف الآلـي عـلى الكـلام.

            نـأتـي بـعـدهـا إلـى تـصـمـيـم نـظام لـتـصـحـيـح الـكـلام الـمـحـتـبـس بـاللـغـة الـفـرنـسـيـة
            يـجـمـع بـيـن نـمـوذجـيـن، أحـدهـمـا للـتـعرف الآلـي عـلى الكـلام والآخـر للـتـرجـمـة الآلـيـة
            ونـعـرض إنـجـاز هـذا الأخـيـر.
            نـخـتـم أخـيـرا بـعـرض نـتـائـج هـذا الـعـمـل مـتـمـثـلـة فـي مـجـمـوعـة بـيـانـات للـتـعرف الآلـي عـلى الكـلام
            ونـمـوذج للـتـرجـمـة الآلـيـة تـقـيـيـمـه بـمـقـيـاس \LR{{bleu}} يـسـاوي \(79.61\%\).
        \end{RLtext}
        \hspace*{0mm}\rule{\linewidth}{1pt}
        \begin{RLtext}
            \textbf{الـكـلـمـات الـمـفـتـاحـيـة ـــ } حـبـسـة بـروكـا،
            تـعـلم الآلـة،
            مـعـالـجـة اللـغـة الـطـبـيـعـيـة،
            تـرجـمـة آلـيـة،
            تـعرف آلـي عـلى الكـلام،
            شـبـكـة عـصـبـيـة غـيـر تـرتـيـبـيـة.\\
        \end{RLtext}
        \rule{\linewidth}{1pt}
    \end{abstract}
    \restoregeometry
}